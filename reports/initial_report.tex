\documentclass[a4paper,12pt]{article}
\usepackage{tgpagella}
\usepackage[T1]{fontenc}
\usepackage{graphicx}
\usepackage{amsmath}
\usepackage{titling}
\usepackage{caption}
\usepackage{newlfont}
\usepackage{cite}
\usepackage{setspace}

\title{LDAP subtree merging proxy daemon}
\author{Ravaioli Michele\thanks{michele.ravaioli3@studio.unibo.it} \and Tampieri Eugenio\thanks{eugenio.tampieri@studio.unibo.it}}

%\textwidth=450pt\oddsidemargin=0pt
\begin{document}
    \maketitle
    \begin{abstract}
        \par We would like to develop an LDAP\footnote{Lightweight Directory Access Protocol, RFC4511}
        proxying daemon (from now on, the frontend) that allows the bridging of multiple LDAP
        servers (the backends) to perform queries and bindings on multiple servers at once. 
        \par The software shall thus forward the requests to the appropriate servers and merge
        the results.
    \end{abstract}
    \section{Requirements and goals}
        \begin{enumerate}
            \item Seamless read-only bridge for LDAP binding and certain types of queries
            \item Cloud native architecture (supports \emph{i.e.} deployment on K8s)
            \item Resilience to server errors implemented by logging and discarding errors
        \end{enumerate}
    \section{Detail of implemented functionalities}
        \subsection{Bind}
            \par When a user binds, the request is forwared according to the DN. The
            connection is then regarded as authenticated for all servers.
        \subsection{Filters}
            \par Basic filters will be supported. However, we will not work around the fact
            that filters requiring referenced attributes will not work across multiple
            directories.
            \par This is because only the server containing the referenced objects will be able
            to access it (for example, if a \texttt{(memberOf=CN=Test,CN=Groups,\\DC=example,DC=com)}
            filter is used, only the query from the server responsible for \texttt{DC=example,DC=com})
            will return a non-empty answer.
    \section{Deliverables}
    \par We plan to deliver an OCI containerised Python application that fulfills the aforementioned
    requirements.
    \par The software shall have a good (\emph{i.e.} $\geq80\%$) test coverage and good configuration
    documentation.
    \section{Work plan}
        \par To achieve the outlined goals and objectives of this project, the following work plan is proposed:
        \begin{enumerate}
            \item Creation of a proper development environment (\emph{i.e.} two or more containerised
            LDAP servers with predefined dummy data). This will be used for unit testing too. (Tampieri)
            \item Studying of the RFCs (Ravaioli)
            \item Implementation of the deliverable (Ravaioli and Tampieri)
            \item Final testing, deployment and documentation (Ravaioli and Tampieri)
        \end{enumerate}
\end{document}
